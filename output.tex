\documentclass{article}

%%%%%%%%%%%%%%%%%
% general styling
% allow us to have more flexible argument lists
\usepackage{xargs}

% alter page layout from default margins/sizes
\usepackage{geometry}
\geometry{
  letterpaper,
  tmargin    = 1.25cm,
  bmargin    = 1.25cm,
  lmargin    = 2.5cm,
  rmargin    = 2.5cm,
  headheight = 0cm,
  headsep    = 0cm,
  footskip   = 0.75cm
}

% change footer; remove page numbering, use fancyhdr (stripping default style
% and removing the header and footer rules)
\pagenumbering{gobble}
\usepackage{fancyhdr} \pagestyle{fancy}
\fancyhf{}
\renewcommand{\headrulewidth}{0cm} \renewcommand{\footrulewidth}{0cm}

% allow us to alter some colors
\usepackage[usenames]{xcolor}

%%%%%%%%%%%%%%%
% CTAN packages
% easily change margins
\usepackage{changepage}

%%%%%%%%%%%%%%
% size styling
\newcommand{\vpad     }{0.50cm}
\newcommand{\thickline}{0.10cm}
\newcommand{\exmargin }{0.75cm}

%%%%%%%%%%%%%%%%%%
% general commands
\newcommand{\mainHeadline}[1]{%
    \addvspace{\vpad}%
    { \noindent \huge #1 \hrulefill \par }%
    \addvspace{\vpad}%
}
\newcommand{\leftHeadline  }[1]{{\Large #1}}
\newcommand{\centerHeadline}[1]{{\it \small #1}}
\newcommand{\rightHeadline }[1]{{\it \large #1 \par}}
\newcommand{\headlineChunk }[1]{\parbox{0.333\textwidth}{#1}}
\newcommand{\subHeadline   }[3]{%
    \noindent%
    \headlineChunk{\raggedright \leftHeadline  {#1}}%
    \headlineChunk{\centering   \centerHeadline{#2}}%
    \headlineChunk{\raggedleft  \rightHeadline {#3}}%
    \par%
}

\newcommand{\stp}{\addvspace{\vpad} \begin{adjustwidth}{\exmargin}{\exmargin}}
\newcommand{\etp}{\end{adjustwidth}\addvspace{\vpad}}
\newenvironment{thinpara}{\stp}{\etp}

%%%%%%%%%%
% sections
\newenvironment{details}{%
    % just store these values as someone fills them out
    \newcommand {\repo   }[1]       {\renewcommand{\REPO   }{##1}        }%
    \newcommand {\name   }[1]       {\renewcommand{\NAME   }{##1}        }%
    \newcommand {\address}[1]       {\renewcommand{\ADDRESS}{##1}        }%
    \newcommand {\phone  }[1]       {\renewcommand{\PHONE  }{##1}        }%
    \newcommandx{\email  }[3][3=com]{\renewcommand{\EMAIL  }{##1@##2.##3}}%
    % special commands to set the footer.
    \newcommand{\setleftfooter  }[1]{\fancyfoot[L]{\color{gray} ##1}}%
    \newcommand{\setcenterfooter}[1]{\fancyfoot[C]{\color{gray} ##1}}%
    \newcommand{\setrightfooter }[1]{\fancyfoot[R]{\color{gray} ##1}}%
    % initialize variables for storage
    \newcommand{\REPO   }{}%
    \newcommand{\NAME   }{}%
    \newcommand{\ADDRESS}{}%
    \newcommand{\PHONE  }{}%
    \newcommand{\EMAIL  }{}%
}{%
    % set the footers - this requires a bit of \expandafter magic because
    % normally you can't access macros inside of \fancyfoot elements, so we've
    % got to pre-expand them.
    \expandafter \setleftfooter   \expandafter{\NAME }%
    \expandafter \setcenterfooter \expandafter{\REPO }%
    \expandafter \setrightfooter  \expandafter{\EMAIL}%
    % output the properly-formatted end results
    \begin{center}%
        % display the info, each element on its own line (name gets its own
        % special heading, with a nice thick line under it)
        { \Huge \NAME \\ \rule{\linewidth}{\thickline} }%
        \ADDRESS \\%
        \PHONE   \\%
        \EMAIL   \\%
    \end{center}%
}

\newenvironment{objective}{\mainHeadline{Objective}\stp}{\etp}

\newenvironment{jobs}{%
    \newenvironment{job}[2]{%
        % just store these values as someone fills them out
        \newcommand{\duration       }[2]{\renewcommand{\CJD}{####1 -- ####2}}%
        \newcommand{\accomplishments}[1]{\renewcommand{\CJF}{####1}}%
        % initialize variables
        \newcommand{\CJT}{##1}% Current Job Title
        \newcommand{\CJD}   {}% Current Job Duration
        \newcommand{\CJC}{##2}% Current Job Company
        \newcommand{\CJF}   {}% Current Job Accomplishments
    }{%
        % actually output the end result after all is said and done
        \subHeadline{\CJT}{\CJD}{\CJC}%
        \begin{thinpara}\CJF\end{thinpara}
    }%
    \mainHeadline{Recent Work History}%
}{}

\newenvironment{skills}{%
    \newenvironment{category}[1]{%
        \newcommand{\skill}[1]{\item ####1}%
        \noindent \leftHeadline{##1} \par%
        \stp \begin{itemize}%
    }{%
        \end{itemize} \etp%
    }%
    \mainHeadline{Skills}%
}{}

\newenvironment{education}{%
    \newenvironment{degree}[2]{%
        \newcommand{\received}[1]{\renewcommand{\CDR}{####1}}%
        % now this gets a little hairy. Every new achievement has to redefine
        % the \CDA macro, so we have to dig into TeX guts a bit here, using
        % \expandafter to force macro expansion *prior* to definitions.
        \newcommand{\achievement}[1]{%
            \expandafter\def\expandafter\CDA\expandafter{\CDA\item{####1}}%
        }%
        % initialize the variable
        \newcommand{\CD }{##1}% Current Degree
        \newcommand{\CDR}   {}% Current Degree Received
        \newcommand{\CDS}{##2}% Current Degree School
        \newcommand{\CDA}   {}% Current Degree Achievements
    }{%
        \subHeadline{\CD}{\CDR}{\CDS}%
        \begin{thinpara}\begin{itemize}\CDA\end{itemize}\end{thinpara}%
    }%
}{}

\begin{document}\sf\begin{details}
    \name{Stephen Belcher}
    % include some dynamically-generated data too.
    \immediate\write18{./output-details.bash}\input{outputs/details.tex}
    \email{sbelcher+jobs}{gmail}
    \repo{https://github.com/sycobuny/resume}
\end{details}

\begin{objective}
    To obtain positions which would enable continued learning, sharing of new
    knowledge, and growth in the technical community, as well as facilitate
    the development of new software and technologies.
\end{objective}

\begin{skills}
    \begin{category}{Computer Languages}
        \skill{%
            TypeScript, JavaScript, and Angular, for building single-page
            applications as well as complex database- and API-driven systems
        }
        \skill{%
            Ruby, including extensive experience with Ruby on Rails,
            Sinatra/Padrino, and wxRuby
        }
        \skill{%
            Perl, including modern web frameworks like Mojolicious, database
            connection with DBI and internal functions in PL/Perl in
            PostgreSQL, familiarity with complex regular expressions, and
            GUI development using the Perl/Tk and wxPerl frameworks
        }
        \skill{%
            Various reporting languages/technologies, such as \LaTeX\ and
            XML-based XLST/XSL-FO dialects
        }
        \skill{%
            PHP, including developing abstract database model systems and
            simplified testing frameworks for producing readable, reliable and
            maintanable code
        }
        \skill{%
            Visual Basic, particularly regarding automated forms in Microsoft
            Office applications
        }
        \skill{%
            Puppet, Java, C/C++, HTML/HAML, CSS/Sass, Vimscript, and Markdown
        }
    \end{category}

    \begin{category}{Database and System Administration}
        \skill{%
            Administration of both Linux and FreeBSD server environments,
            including scheduling tasks with cron, setting up web server and
            database server daemon processes, and management of software
            packages and utilities
        }
        \skill{%
            Setting up and managing systems using Puppet, including developing
            custom modules, types, providers, and functions in native Ruby
        }
        \skill{%
            Administration and upgrades for local GitLab instances for private
            collaborative coding
        }
        \skill{%
            PostgreSQL administration experience, including designing custom
            datatypes, functions, triggers, views, and server tuning and
            creation of automatic management and monitoring scripts
        }
        \skill{%
            Experience designing and implementing databases in MySQL and
            SQLite
        }
        \skill{%
            Familiarity with accessing and working with multiple legacy
            systems, including Microsoft Access and FoxPro
        }
    \end{category}

    \begin{category}{Other}
        \skill{%
            Frequent technical speaker at technical meetings and conferences,
            with clear and concise talks on various topics
        }
        \skill{%
            Host of the Baltimore Vim user group, holding monthly meetings
            with regular talks by local professionals
        }
        \skill{%
            Volunteer organizer for the DC/Baltimore Perl Workshop 2014--2016
        }
        \skill{%
            Volunteer organizer and SOC Chair for the 2017 Perl Conference in
            Alexandria, VA
        }
    \end{category}
\end{skills}

\begin{jobs}
    \begin{job}{Programmer}{Kelly Services, Inc.}
        \duration{December 2011}{Present}
        \accomplishments{%
            Responsible for maintaining large-scale existing web applications
            using various languages, such as Ruby, Python, PHP and Perl, and
            database engines, such as PostgreSQL and MySQL\@. This includes
            fixing bugs, patching security holes, as well as adding new
            feature requests from users.

            Designed multiple new websites from the ground up, including
            meeting with stakeholders to assess requirements, create work
            estimates, and develop using modern practices to ensure ease of
            future maintenance.

            Fostered a culture of good practices, including use of revision
            control for projects and development of open-souce tools to
            increase re-usability of software. Also encouraged strong
            foundational security standards to guarantee applications are
            written more securely from the outset, ensuring greater stability
            overall.

            Maintained systems using Puppet with custom-written modules,
            types, and classes, and managed GitLab along with GitLab-CI for
            more productive development.
        }
    \end{job}

    \begin{job}{Database Consultant}{EBL Engineers, LLC}
        \duration{December 2011}{Present}
        \accomplishments{
            Maintenance of existing database and application infrastructure
            for continued use, including security updates and maintaining
            backup scripts.
        }
    \end{job}

    \begin{job}{Database Manager}{EBL Engineers, LLC}
        \duration{June 2005}{December 2011}
        \accomplishments{%
            Work included creating applications for use in maintaining
            hospital accreditation according to continually-updated standards
            from The Joint Commission in service of the Clinical Research
            Center of the National Institutes of Health.

            Designed a database to represent a hospital facility, its
            operations and staff, including ongoing construction projects,
            equipment, and periodic maintenance. The database includes the
            ability to monitor the facility's temporal evolution through time,
            taking into account modification of physical space over time
            through construction projects.

            Designed and implemented various interfaces to the database,
            according to the evolving needs of the organization. These include
            server APIs, web interfaces, cross-platform GUI controls, and
            automatic reporting to CSV, PDF, and other formats.

            Hired and managed a system administrator/programmer to decrease
            time to develop deliverables and enhance the quality of finished
            work.  Worked with the system administrator in developing a set of
            security controls for government standards compliance.
        }
    \end{job}
\end{jobs}

\begin{education}
    \begin{degree}{Bachelor of Music}{Peabody Conservatory}
        \received{May 2005}
        \achievement{%
            Studied computer science at parent university Johns Hopkins,
            including coursework on programming, data structures, and
            CPU/Assembler fundamentals
        }
        \achievement{Dean's List}
    \end{degree}

    \begin{degree}{High School Diploma}{Baltimore School for the Arts}
        \received{May 2002}
        \achievement{%
            Member of the Daniel Ramos Chapter of the National Honor Society
        }
    \end{degree}
\end{education}
\end{document}